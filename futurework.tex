\chapter{Future Work}
\label{future}

The formalisation of Lyapunov’s Central Limit Theorem (CLT) in this thesis is nearly complete. All analytic components—including moment bounds, Taylor expansion, auxiliary sequences, and asymptotic estimates—have been mechanised. The final convergence theorem has been stated and is fully supported by the developed infrastructure. The only remaining step is to formalise the third absolute moment of the Gaussian distribution:

\[
\mathbb{E}[|X|^3] = \sqrt{8/\pi} \cdot \sigma^3.
\]

This requires formalising improper integrals and special functions, particularly the gamma function. Currently, HOL4 lacks sufficient support for such techniques. Bridging the gap between classical Riemann-style and formal Lebesgue integration will be critical for validating these results. The author intends to address this in future work.

Beyond completing this proof, several directions arise naturally:

\subsection*{1. Generalising the Central Limit Theorem}

The current version handles independent, non-identically distributed variables. Future formal developments could include:

\begin{itemize}
\item \textbf{CLT for Martingales.} Extend the current proof to martingale difference sequences, which generalise independence to adapted stochastic processes \cite{hall2014martingale}.
\item \textbf{CLT under Weak Dependence.} Formalise convergence under weak dependence or mixing conditions, as outlined in Billingsley~\cite{billingsley2017probability}.
\item \textbf{Multivariate CLT.} Extend the proof to sequences of random vectors in \( \mathbb{R}^d \), a standard result in multivariate statistics.
\item \textbf{Berry–Esseen Bounds.} Provide a quantitative rate of convergence, leveraging the moment bounds and error control machinery developed in this thesis \cite{berry1941accuracy}.
\end{itemize}

\subsection*{2. Extending HOL4's Mathematical Libraries}

The formalisation also highlights areas where HOL4 could be extended to support more advanced probability theory:

\begin{itemize}
  \item Definitions and properties of the gamma, beta, and error functions.
  \item Support for improper Lebesgue integration on infinite domains.
  \item Formalisation of complex-valued functions and Fourier/characteristic methods.
  \item Improved automation for Big-O and asymptotic reasoning in formal proofs.
\end{itemize}

These enhancements would enable broader mechanisation of probability theory, from classical results to modern topics in stochastic processes and statistics. The infrastructure developed in this thesis demonstrates the feasibility of such mechanisation and provides a reusable foundation for further research.
