\chapter{Conclusion}
\label{concl}

This thesis has formalised a general version of the Central Limit Theorem (CLT) in HOL4 under Lyapunov's condition, using the Lindeberg replacement method. The version presented here is far more general than the classical I.I.D.~case: it handles independent (but not identically distributed) summands with varying variances and third absolute moments. Our development closely follows Chung's analytical presentation~\cite{Chung:2001}, and surpasses earlier formal work such as the CLT in Isabelle/HOL~\cite{serafin2015formally} in both generality and technical depth.

To the best of our knowledge, this is the first mechanised proof of CLT beyond the I.I.D.~cases.
This version, usually known as the \emph{Lyapunov's form of the CLT}, employs the Lindeberg's method of replacement trick, Taylor expansion, and Big-O reasoning. Unlike earlier approaches that assume identical distributions, our formalisation accommodates inhomogeneous sequences, which introduces substantial technical overhead--particularly in bounding variables--specific replacement errors. However, this effort yields a more widely applicable theorem under Lyapunov’s condition.

Over 6000 lines of HOL4 proof scripts were developed to support this formalisation, including infrastructure for moment inequalities, summability, Taylor approximation, and asymptotic analysis. Much of this was written from scratch or adapted from existing theories to suit the strict requirements of probabilistic reasoning in HOL4. The resulting tools significantly enhance HOL4’s capacity to reason about convergence in distribution and limit theorems.

Initial attempts to formalise the CLT via moment-generating functions proved impractical due to the lack of support for improper integrals and MGFs in HOL4’s libraries. The switch to the Lindeberg method, though more technically involved, allowed for a successful proof strategy grounded entirely in measure theory and real analysis.

\medskip

\textbf{What has been completed:}
\begin{itemize}
    \item Construction of auxiliary sequences of Gaussian (normal) random variables;
    \item Moment and variance control via Lyapunov-type inequalities;
    \item Taylor expansion bounds on individual replacement steps;
    \item Big-O asymptotic estimation of the total error.
\end{itemize}

\textbf{What remains:}
The only remaining work is to complete the proof of the final convergence statement. Specifically, it entails formalising the absolute third moment of the normal distribution as an integral expressed in terms of the gamma function. This requires evaluating \( \mathbb{E}[|X|^3] \) for \( X \sim \mathcal{N}(0, \sigma^2) \), which involves improper integrals over \( \mathbb{R} \), as well as properties of special functions such as \( \Gamma(z) \). Unfortunately, the necessary infrastructure for complex-valued integration, special functions, and improper integrals is not yet available in the standard HOL4 libraries.

Beyond the lack of special-function support, a deeper gap remains between classical Riemann-based probabilistic analysis and the rigorous Lebesgue-style formalism adopted in HOL4. Many standard results rely on transitions between these views, which must be made fully explicit in a mechanised setting. Bridging this divide—by relating improper Riemann and Lebesgue integrals—is essential for validating textbook-level probabilistic computations, and forms an important direction for future work.

At the time of writing, all analytic components have been formalised, and the final error bound has been cleanly isolated. The only missing piece is the formalisation of the third moment of the Gaussian, a purely technical step requiring extended integration support in HOL4.

\medskip

\noindent
This work lays a solid foundation for future research on formalising limit theorems in probability theory. As support for complex integration and special functions in HOL4 grows, the full proof of the CLT will become attainable. Beyond that, the techniques developed here could be extended to CLTs for dependent sequences, martingales, or even random vectors. This thesis not only demonstrates the feasibility of formalising one of probability theory’s central results but also provides reusable infrastructure for reasoning about expectations, variances, Taylor approximations, and convergence in distribution within a theorem prover.
