%%% -*- Mode: LaTeX; -*-
\chapter{Conclusion}
\label{concl}

We have formalized in this thesis, to a great extent, a general version of the Central Limit Theorem (CLT) in HOL4 under Lyapunov's condition, with the Lindeberg replacement method. The version of the CLT proved herein is far more general than the classical i.i.d. case: it treats summands which are independent but not necessarily identically distributed, with variances and third absolute moments which are not necessarily constant over indices. This is in line with Chung's analytical development \cite{Chung:2001} and is a much stronger version of the theorem than those formalised in theorem provers such as Isabelle/HOL previously \cite{serafin2015formally}.

To our knowledge, this is the first mechanised proof of Lyapunov’s CLT under general independent sequences using Lindeberg replacement method, Taylor expansion and Big-O bounds in HOL4. Compared to the CLT formalized in Isabelle, which assumes the same distribution, our presentation deals with the additional challenge of summands being inhomogeneous. This introduces some technical awkwardness, especially in managing variable-specific error terms in Lindeberg replacement. The result is a version of the CLT valid under Lyapunov's condition—a broader and more applicable class—at the cost of much more intricate formal proofs.

The work involved developing over 6000 lines of HOL4 code, from moment inequalities, summability, Taylor approximation, and asymptotic Big-O estimation theorems and definitions. Most of this infrastructure either had to be developed from scratch or redeveloped to suit the requirements of probability formalisation in HOL4. The development significantly extends the scope of HOL4 for conducting convergence in distribution and limit theorem reasoning.

Early progress on the project was hindered by the learning curve of HOL4 and an initial try at proving the CLT through moment-generating functions. Even though this path is textbook typical, it was unworkable in HOL4 since nothing on MGFs and improper integrals was available in the standard libraries. The transition to the Lindeberg substitute method, while more awkward, was successful and allowed all main analytic components of the CLT to be formally rigorised.

\medskip

\textbf{What has been completed:}
\begin{itemize}
    \item Construction of auxiliary sequences of Gaussian random variables;
    \item Moment and variance control by Lyapunov-type inequalities;
    \item Taylor expansion for the sum and remainder bounds;
\item Big-O asymptotic bounds on the total Taylor approximation error.
\end{itemize}

\textbf{What remains:}
The only remaining work is to complete the proof of the final convergence statement. Specifically, it entails formalising the absolute third moment of the normal distribution as an integral expressed in terms of the gamma function. This requires evaluating \( \mathbb{E}[|X|^3] \) for \( X \sim \mathcal{N}(0, \sigma^2) \), which involves improper integrals over \( \mathbb{R} \), as well as properties of special functions such as \( \Gamma(z) \). Unfortunately, the necessary infrastructure for complex-valued integration, special functions, and improper integrals is not yet available in the standard HOL4 libraries.

In addition to the lack of special function formalisation, a more foundational gap remains between the classical Gauss–Riemann-style integration used in traditional probability theory and the fully formal Lebesgue integration developed in HOL4. Many textbook results rely on informal transitions between these two views of integration, which in a formal setting must be rigorously connected. Bridging this gap—formally relating improper Riemann integrals to their Lebesgue counterparts—is essential for justifying many classical computations, and is currently an area of planned future work.

At the time of writing, the final error bound has been cleanly isolated and stated. All supporting lemmas, inequalities, and approximation steps are in place. The remaining step is therefore not conceptual but purely technical: implementing the final integration using extended analysis tools once they become available in HOL4.

\medskip

\noindent
This thesis gives a good basis for further research work on the formalisation of limit theorems in probability theory. As support for complex integration and special functions in HOL4 evolves, it will be possible to complete the whole proof and perhaps extend to CLTs for dependent variables, martingales, or random vectors. The work presented here not only demonstrates that one can mechanise one of the classic results in probability theory but also develops reusable infrastructure for formalising probabilistic proofs with expectations, variances, Taylor approximations, and convergence in distribution.
