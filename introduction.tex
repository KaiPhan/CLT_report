%%% -*- Mode: LaTeX; -*-
\chapter[Introduction]{Introduction}
\label{chap:1}

The idea of a formalised mechanist is simple but powerful: if we want to trust mathematics in software, we must mechanise not only calculations but also proofs. A mechanist builds a logic engine where every theorem arises from a small set of primitive rules. If those rules are sound, then everything derived from them inherits that soundness.

HOL4 supports this mechanised philosophy~\cite{slind2008brief}. It encodes logic using a typed \(\lambda\)-calculus and enforces soundness by restricting theorem creation to a protected MetaLanguage (ML) type. The only way to construct a theorem is by applying primitive inference rules written in Standard ML. This ensures that the proof kernel checks every line of reasoning with absolute rigour.

In this thesis, we formalise one of the most important results in probability theory: the Central Limit Theorem (CLT). When many independent effects accumulate, their combined influence often resembles a bell curve. We observe this phenomenon across a wide range of domains—from biological traits and human measurements to economic indicators and communication errors. The CLT explains this universality.

We follow the version of the Central Limit Theorem stated in~\cite{chung2000course} and proved using Lindeberg’s method. Let $\{X_j\}_{j=1}^n$ be a sequence of independent random variables such that:

$$
\mathbb{E}[X_j] = 0,\quad \mathbb{E}[X_j^2] = \sigma_j^2,\quad \mathbb{E}[|X_j|^3] = \gamma_j < \infty.
$$

Define the sum $S_n = \sum_{j=1}^n X_j$, the total variance $s_n^2 = \sum_{j=1}^n \sigma_j^2$, and the total third moment $\Gamma_n = \sum_{j=1}^n \gamma_j$. If the Lyapunov condition

$$
\lim_{n \to \infty} \frac{\Gamma_n}{s_n^3} = 0
$$

holds, then the normalised sum \(\frac{S_n}{s_n}\) converges in distribution to the standard normal:

$$
\frac{S_n}{s_n} \xrightarrow{d} \mathcal{N}(0, 1).
$$

This convergence result explains why the normal distribution emerges from a wide variety of processes, even when individual random variables follow skewed or irregular distributions.

We do not assume probability theory or integration as axioms. Instead, we construct them from first principles in HOL4 using its existing real number, measure theory, and Lebesgue integration libraries. We define random variables, convergence in distribution, and weak limits in this framework. We then prove the CLT using the Lindeberg replacement method and Taylor expansion bounds, following the structure outlined in~\cite{Chung:2001}.
Here is the Lyapunov's form of the Central Limit Theorem in HOL4:
\begin{hol}
  \begin{alltt}
    Theorem central_limit_theorem :
    \(\!\!\!{\turn}\!\!\!\!\) \(\forall\)p X N. prob_space p \(\land\)
    ext_normal_rv N p 0 1 \(\land\)
    (\(\forall\)i. real_random_variable (X i) p) \(\land\)
    (\(\forall\)n. indep_vars p X (\(\lambda\)i. Borel) (count n)) \(\land\)
    (\(\forall\)i. expectation p (X i) = 0) \(\land\)
    (\(\forall\)i. expectation p (\(\lambda\)x. (abs (X i x)) pow 3) < +\(\infty\)) \(\land\)
    (\(\forall\)i. variance p (X i) +\(\infty\)) \(\land\)
    (\(\forall\)i. variance p (X i) \(\ne\) 0) \(\land\)
    (\(\forall\)n. (sqrt (second_moments p X n)) \(\ne\) 0) \(\land\)
    ((\(\lambda\)n. (third_moments p X n) / ((sqrt (second_moments p X n)) pow 3)) --> 0)
    sequentially \({\Leftrightarrow}\)
    ((\(\lambda\)n x. (SIGMA (\(\lambda\)i. X i x) (count n)) / (sqrt (second_moments p X n))) --> N)
    (in_distribution p)
  \end{alltt}
\end{hol}


%-----------------------------------------------------

%This formalisation produces a complete, machine-checked proof of Lyapunov’s Central Limit Theorem. It not only demonstrates the power of HOL4 for probabilistic reasoning, but also enriches its measure-theoretic library with tools necessary for formalising deeper results in probability and statistics.
%\newline
%-----------------------------------------------------

Despite not yet completing the final convergence proof, this thesis represents a significant step toward a fully mechanised proof of Lyapunov’s Central Limit Theorem in HOL4. The author developed over 6000 lines of formalisation code, building most of the necessary measure-theoretic and probabilistic infrastructure. Early progress was slowed by the need to learn the HOL4 system and an initial attempt to formalise the CLT using moment-generating functions—a path that proved impractical due to missing theorems in the HOL4 library at the time. The current approach, based on Lindeberg's method and Taylor expansion, has succeeded in formalising all major components of the proof. The only remaining step—formalising the Taylor error bound  and completing the convergence argument—involves calculating the absolute third moment of the normal distribution. This means formalizing integrals using the gamma function and complex numbers, which are currently not yet fully realized in HOL4. At the time of writing the report, the author is still in the midst of this last step. Even if it is still incomplete at the submission deadline, the overall structure and supporting lemmas are all in place, and the last bound is now a question of careful implementation instead of conceptual difficulty.

\section{Motivation}
The Central Limit Theorem (CLT) is more than just a result in probability theory—it is the mathematical explanation behind the bell curve that appears across nature, science, and data. When we take a large number of independent random variables with finite variance, their properly normalised sum tends toward a normal distribution, regardless of the original distributions of the individual variables~\cite{fischer2011history}.This effect manifests in physical measurements, biological readings, economic results, and even in computation roundoff errors. The CLT illustrates the reasons why such behavior is not accidental but is necessary mathematically under gentle hypotheses.

Although succinctly phrased, the CLT is of very deep mathematical organization. Most textbook proofs rely on informal approximations or skip foundational details. In formal verification, this level of vagueness is unacceptable. We need a version of the CLT where every assumption is explicit, every step justified, and the entire argument checked by machine.

Some existing formal proofs of the CLT, such as the one in Isabelle/HOL~\cite{serafin2015formally}, focus on the independent identical distribution (i.i.d.) case and rely on characteristic functions and the Lévy Continuity Theorem. That approach is elegant, but it depends on complex analysis and Fourier transforms, and does not generalise easily to sums of non-identically distributed variables. Moreover, HOL4’s complex number library is still underdeveloped and lacks many of the theorems needed to support such an approach effectively. Attempting to formalise the necessary complex theory would demand substantial additional effort—likely more than what is needed for the Lindeberg-based approach we follow in this work.

In this thesis, we aim for something different. We want to formalise a version of the CLT that:
\begin{itemize}
\item allows non-identically distributed variables;
\item avoids characteristic functions entirely;
\item uses only measure-theoretic and analytical tools available in HOL4.
\end{itemize}

We follow the proof given in~\cite{Chung:2001}, based on Lindeberg’s replacement trick and Taylor expansion bounds. This approach aligns closely with classical analysis and is modular enough to support future generalisations—such as multidimensional CLTs or stable laws.

By formalising this proof in HOL4, we not only demonstrate that the CLT is derivable in a foundational logical system, but also contribute tools and theorems that strengthen HOL4’s probability theory library for future formalisation work.

\section{Thesis Contributions}
The contribution of this work is the formalisation of Lyapunov's form of the Central Limit Theorem in HOL4, using the Lindeberg replacement strategy and Taylor approximation techniques. This required building up the necessary infrastructure for real-valued random variables, expectations, variances, and convergence in distribution from the ground up.

In particular:

\begin{itemize}
\item We constructed the formal setting for expressing the CLT in Lyapunov form, augmenting HOL4 with normalisation tools, moment summation tools, and standardisation tools.
\item We established the background required to define and establish bounds on expectations using Taylor expansion, including a remainder term of order three.
\item We proved key lemmas for the Lindeberg substitution trick, which summarizes how to replace original variables by Gaussian approximations and bound resulting error.
\item We defined the final convergence goal in terms of convergence in distribution with a well-defined formulation and separation of technical specifications.
\end{itemize}

\section{Structure of the Thesis}
The remainder of this thesis is organised as follows. Chapter 2 reviews the Central Limit Theorem, its historical development, and the version we aim to formalise. It also surveys related work, including formal proofs in systems such as Isabelle/HOL. Chapter 3 introduces the HOL4 theorem prover and outlines the relevant libraries and foundational setup. Chapter 4 presents the core of the formalisation, including the supporting lemmas and structure of the proof. Chapter 5 discusses the results and places this work in the context of other efforts in formalised mathematics.Chapter 6 summarises the thesis by stating the contributions and providing guidance for future development of probability theory in HOL4. This research proves that a formalisation of the Central Limit Theorem in HOL4 is possible. It enriches HOL4's probability library, provides reusable tools for future developments, and establishes the system as a platform for fully verified mathematics.
