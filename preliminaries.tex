\chapter[Preliminaries]{Preliminaries}
\label{chap:3}

This chapter describes an overview of the theoretical and formal theories required for the formalization of the Central Limit Theorem . This includes HOL Formalization, Measure Theory, Lebesgue Integration, and Probability Theory.
\section{HOL Formalization}
Higher Order Logic (HOL) \cite{hol4, slind2008brief} is derived from the Logic of Computable Functions (LCF) \cite{gordon1979edinburgh, milner1972logic} created by Robin Milner and colleagues in 1972. HOL is an adaptation of Church's Simple Theory of Types (STT) \cite{church1940formulation}, where a higher-order version of Hilbert's choice operator $\varepsilon $, Axiom of Infinity, and Rank-1 polymorphism have been added. HOL4 implements the original HOL framework, while other theorem provers in the HOL family, such as Isabelle/HOL, include important extensions. Such a simple logical basis makes HOL more accessible than those systems founded on much more advanced dependent type theories, such such the Calculus of Inductive and Co-Inductive Constructions constructed by Coq. Therefore, theories and proofs founded on HOL are easier for a layman to comprehend rather than being lost in a complicated type theory.

HOL refers both to the logical system and the software implementing it. HOL4 is the latest version of this software and written in Standard ML (SML), a general-purpose functional programming language.  SML has played the most vital role in the HOL4 for implementing its core engine, enabled automation due to which proof tactics have been written in that and also for interaction, whether it is through a proof script or in direct correspondence with the user. Integrated SML gives a way in which HOL4 is versatile and can easily be extended such that complex verification tools are provided to develop the management of proofs by a user efficiently.

The type system of HOL establishes the structural framework within which all terms and expressions are guaranteed to be well-defined and logically consistent. Types in HOL denote sets within the universe $ U $, and every term bears a certain type. The type grammar is simple and very expressive, and thus able to construct a wide variety of mathematical and logical objects.

The type grammar is defined as:
\[
\sigma ::= \alpha \ | \ c \ | \ (\sigma_1, \ldots, \sigma_n)\text{op} \ | \ \sigma_1 \to \sigma_2
\]
where:
\begin{enumerate}
    \item \textbf{Type Variables ($\alpha, \beta, \ldots$):} Generic placeholders that allow polymorphism to provide functions and predicates over different types.
    \begin{itemize}
        \item Example: The type variable $\alpha$ could indicate integers, Booleans, or functions.
    \end{itemize}

    \item \textbf{Atomic Types ($c$):} Fixed and pre-defined types within HOL. The two initial atomic types are:
    \begin{itemize}
        \item \texttt{bool}: The set of Boolean values $\{T, F\}$.
        \item \texttt{ind}: The set composed by individuals (an infinite set).

    \end{itemize}

    \item \textbf{Compound Types ($(\sigma_1, \ldots, \sigma_n)\text{op}$):} Formed by applying type operators to other types. Their examples include Cartesian products, which designate the tuples over the elements.
    \begin{itemize}
        \item Example: The type $(\texttt{bool}, \texttt{ind}) \times$ represents pairs of a Boolean and an individual.
    \end{itemize}

    \item \textbf{Function Types ($\sigma_1 \to \sigma_2$):} Represent total functions mapping elements from a domain ($\sigma_1$) to a codomain ($\sigma_2$).
    \begin{itemize}
        \item Example: The type $\texttt{bool} \to \texttt{ind}$ indicates a function mapping both Boolean-values to individual-elements.
    \end{itemize}
\end{enumerate}

For example, consider the following types:
\begin{enumerate}
    \item A function from integers to Booleans:
   \[
   f : \texttt{int} \to \texttt{bool}
   \]
   This type indicates that $f(x)$ is a function taking an integer $x$ and returning a Boolean.

    \item A tuple containing a Boolean and a function:
       \[
       p : (\texttt{bool}, (\texttt{int} \to \texttt{bool}))
       \]
       This is a pair type $p = (b, f)$, where $b$ is a Boolean, and $f$ is a function mapping integers to Booleans.
       \item The type system guarantees the consistency by making sure all terms are properly typed. So if $g : \texttt{int} \to \texttt{bool}$, then $g(5)$ as 5 is an integer, but, $g(T)$ would be invalid since $T$ is a Boolean, not an integer. Such stringent typing is at the level of terms to avoid self-contradictory values and assure that proofs built up in HOL are sound.
\end{enumerate}

In HOL, terms are representatives for elements of sets represented by their types. The grammar of the term defines the syntax and structure for the logical expressions that can be expressed and hence statements that could be well typed and logically valid. Terms in HOL are constructed from the following components:

\[
t ::= x \ | \ c \ | \ t \ t' \ | \ \lambda x. t
\]

where:
\begin{enumerate}
    \item \textbf{Variables} $(x, y, \ldots):$
    \begin{itemize}
        \item Represent placeholders for elements of a type.
        \item Example: $x : \texttt{bool}$ stands for a Boolean variable.
    \end{itemize}

    \item \textbf{Constants ($c$):}
    \begin{itemize}
        \item Fixed entities such as $T$, $F$, mathematical operators, or predefined functions.
        \item  Example: The constant $+$ defines addition for numeric types.
    \end{itemize}

   \item \textbf{Function Applications ($t \ t'$):}
   \begin{itemize}
       \item Define the application of a function to an argument. The term $f(x)$ applies the function $f$ to the variable $x$.
        \item Example: If $f : \texttt{int} \to \texttt{real}$ and $x : \texttt{int}$, then $f(x)$ is a valid term of type $\texttt{real}$.
   \end{itemize}

\item \textbf{$\lambda$-Abstractions ($\lambda x. t$):}
\begin{itemize}
    \item Denote anonymous functions where $x$ is the input variable, and $t$ is the function body.
    \item Example: $\lambda x. x + 1$ defines a function that increments its input by 1.
\end{itemize}


\end{enumerate}

To ensure consistency, the terms of HOL should be well typed. Given a term $t_{\sigma}$ of type $\sigma$, its grammar can be generalized with type annotations:

\[
t_{\sigma} ::= x_{\sigma} \ | \ c_{\sigma} \ | \ (t_{\sigma_1 \to \sigma_2} \ t'_{\sigma_1})_{\sigma_2} \ | \ (\lambda x_{\sigma_1}. t_{\sigma_2})_{\sigma_1 \to \sigma_2}
\]

HOL's deductive system is considered the logical foundation for forming and checking a proof. HOL's deductive system may consist of eight primitive rules of inference, the definition of new theorems by existing theorems. These rules are the basic components and are required for all logical reasoning within HOL, ensuring that proofs are consistent, logically valid, and traced. The following are the eight main primitive inference rules in HOL:

\begin{enumerate}
    \item \textbf{Assumption Introduction (ASSUME):}
    \begin{itemize}
        \item Introduces a formula as an assumption.
        \item Rule:
        \item Example: From the assumption $P$, we conclude $P$.
    \end{itemize}

    \item \textbf{Reflexivity (REFL):}
    \begin{itemize}
        \item States that any term is equal to itself.
        \item Rule:
        \item Example: For $x : \texttt{int}$, $x = x$ is always true.
    \end{itemize}

    \item \textbf{Beta Conversion (BETA\_CONV):}
    \begin{itemize}
        \item Applies substitution in lambda abstractions.
        \item Rule:
        \item Example: $(\lambda x. x + 1)(5) \vdash 5 + 1$.
    \end{itemize}

    \item \textbf{Substitution (SUBST):}
    \begin{itemize}
        \item Replaces a term in a formula with another term proven to be equal.
        \item Rule:
        \item Example: From $x = y$ and $P(x)$, infer $P(y)$.
    \end{itemize}

    \item \textbf{Abstraction (ABS):}
    \begin{itemize}
        \item Generalizes an equation by abstracting a variable.
        \item Rule:
        \item Example: From $5 + 1 = 6$, infer $\lambda x. x + 1 = \lambda x. 6$.
    \end{itemize}

    \item \textbf{Type Instantiation (INST\_TYPE):}
    \begin{itemize}
        \item Specializes polymorphic functions or predicates to specific types.
        \item Rule:
        \item Example:
    \end{itemize}

    \item \textbf{Discharging Assumptions (DISCH):}
    \begin{itemize}
        \item Converts an assumption into an implication.
        \item Rule:
        \item Example: From $P \wedge Q$, infer $P \Rightarrow (Q \wedge P)$.
    \end{itemize}

    \item \textbf{Modus Ponens (MP):}
    \begin{itemize}
        \item Combines an implication and its premise to infer the conclusion.
        \item Rule:
        \item Example: From $x > 0 \Rightarrow x^2 > 0$ and $x > 0$, infer $x^2 > 0$.
    \end{itemize}
\end{enumerate}
These inference rules ensure that all logical derivations are traceable to basic axioms and established theorems. Additionally, the deductive system forms the backbone of HOL4, ensuring that proofs are both rigorous and reliable.

All proofs in HOL are fundamentally derived from a set of primitive inference rules and a core logical foundation. These rules define the semantics of two fundamental logical connectives: \textbf{equality} ($=$) and \textbf{implication} ($\implies$). Other logical connectives and first\-order quantifiers, such as logical truth ($T$), falsehood ($F$), conjunction ($\wedge$), disjunction ($\vee$), and existential quantification ($\exists$), are defined as lambda ($\lambda$) functions for consistency within the HOL framework:

\begin{enumerate}
    \item \textbf{Logical Truth ($T$)}
    \begin{itemize}
        \item Rule
        \item True is represented as the equality of two identical boolean functions.
    \end{itemize}

    \item \textbf{Logical Falsehood ($F$)}
    \begin{itemize}
        \item Rule
        \item False is defined to satisfy any boolean implication.
    \end{itemize}

    \item \textbf{Negation ($\neg$)}
    \begin{itemize}
        \item Rule
        \item Negation is the implication of a boolean value leading to falsehood.
    \end{itemize}

    \item \textbf{Conjunction ($\wedge$)}
    \begin{itemize}
        \item Rule
        \item Conjunction is defined as a logical function that evaluates nested implications.
    \end{itemize}

    \item \textbf{Conjunction ($\wedge$)}
    \begin{itemize}
        \item Rule
        \item Conjunction is defined as a logical function that evaluates nested implications.
    \end{itemize}

    \item \textbf{Disjunction ($\vee$)}
    \begin{itemize}
        \item Rule
        \item Disjunction is expressed through sequential implications.
    \end{itemize}

    \item \textbf{Universal Quantifier ($\forall$)}
    \begin{itemize}
        \item Rule
        \item Universality asserts that a predicate holds for all elements of a type.
    \end{itemize}

    \item \textbf{Existential Quantifier ($\exists$)}
    \begin{itemize}
        \item Rule
        \item Existence is defined using Hilbert’s choice operator ($\varepsilon$).
    \end{itemize}

HOL also defines constructs for mathematical operations, such as \textbf{one-to-one functions} ($One\_One$) and \textbf{onto functions} ($Onto$), to extend logical capabilities:

    \item \textbf{One-to-One ($One\_One$)}
    \begin{itemize}
        \item Rule
    \end{itemize}

    \item \textbf{Onto ($Onto$)}
    \begin{itemize}
        \item Rule
    \end{itemize}

HOL includes the constant $Type\_Definition$, which defines new types as bijections of subsets of existing types:
    \item \textbf{Type Definition $Type\_Definition$}
    \begin{itemize}
        \item Rule
        \item This process is automated by the HOL Datatype package, simplifying the creation of new types.

    \end{itemize}


\end{enumerate}

HOL’s standard theory is built upon four foundational axioms:
\begin{enumerate}
    \item \textbf{Boolean Cases (BOOL\_CASES\_AX)}
    \begin{itemize}
        \item Rule
        \item This axiom ensures that any boolean value is either true or false.
    \end{itemize}

    \item \textbf{Eta Conversion (ETA\_AX)}
    \begin{itemize}
        \item Rule
        \item Eta conversion describes the extensionality of functions.
    \end{itemize}

    \item \textbf{Hilbert’s Choice (SELECT\_AX)}
    \begin{itemize}
        \item Rule
        \item This axiom relates the choice operator to existential quantification.
    \end{itemize}

    \item \textbf{Infinity (INFINITY\_AX)}
    \begin{itemize}
        \item Rule
        \item The Axiom of Infinity ensures the existence of an infinite set.
    \end{itemize}

\end{enumerate}

These axioms are generally sufficient for conventional formalization projects in HOL4. Adding new axioms is strongly discouraged, as it can compromise logical consistency.

\section{Measure Theory}

Measure theory provides the mathematical foundation for probability and integration. In this project, we rely on the HOL4 formalisation of measure spaces, $\sigma$-algebras, and measurable functions, as defined in the `measureTheory` library \cite{hol4}.

\subsection{Measure Spaces}

A \emph{measure space} is a triple $ (X, \Sigma, \mu) $, where:
\begin{itemize}
  \item $ X $ is the underlying space,
  \item $ \Sigma $ is a $ \sigma $-algebra of subsets of $ X $,
  \item $ \mu : \Sigma \to \overline{\mathbb{R}} $ is a measure.
\end{itemize}

In HOL4, these components are represented as:
\begin{hol}
  \begin{alltt}
    Definition m\_space\_def :
    \(\!\!\!{\turn}\!\!\!\!\) \(\forall\)sp sts \(\mu\).
    m\_space (sp, sts, \(\mu\)) = sp
  \end{alltt}
\end{hol}

\begin{hol}
  \begin{alltt}
    Definition measurable\_sets\_def :
    \(\!\!\!{\turn}\!\!\!\!\) \(\forall\)sp sts \(\mu\).
    measurable\_sets (sp, sts, \(\mu\)) = sts
  \end{alltt}
\end{hol}

\begin{hol}
  \begin{alltt}
    Definition measure\_def :
    \(\!\!\!{\turn}\!\!\!\!\) \(\forall\)sp sts \(\mu\).
    measure (sp, sts, \(\mu\)) = \(\mu\)
  \end{alltt}
\end{hol}


A measure $ \mu $ satisfies:
\begin{enumerate}
  \item \textbf{Non-negativity:} $ \mu(A) \geq 0 $ for all $ A \in \Sigma $,
  \item \textbf{Null empty set:} $ \mu(\emptyset) = 0 $,
  \item \textbf{Countable additivity:} For disjoint sets $ \{A_i\} \subseteq \Sigma $,
  \[
    \mu\left(\bigcup_{i=1}^\infty A_i\right) = \sum_{i=1}^\infty \mu(A_i).
  \]
\end{enumerate}

HOL4 uses the extended real number type \texttt{extreal}, which includes $ +\infty $, $ -\infty $, and undefined values, to represent measure values.

\subsection{$\sigma$-Algebras}

A $\sigma$-algebra $ \Sigma $ over a set $ X $ is a collection of subsets satisfying:
\begin{itemize}
  \item $ X \in \Sigma $,
  \item If $ A \in \Sigma $, then $ X \setminus A \in \Sigma $,
  \item If $ A_i \in \Sigma $ for all $ i \in \mathbb{N} $, then $ \bigcup_i A_i \in \Sigma $.
\end{itemize}

\begin{hol}
  \begin{alltt}
    Definition sigma\_algebra\_def :
    \(\!\!\!{\turn}\!\!\!\!\) \(\forall\)a.
    sigma\_algebra a \(\Leftrightarrow\)
    algebra a \(\land\)
    (\(\forall\)c. countable c \(\land\) c \(\subseteq\) subsets a \(\Rightarrow\) BIGUNION c \(\in\) subsets a)
  \end{alltt}
\end{hol}


\subsection{Measurable Functions}

A function $ f : X \to Y $ is \emph{measurable} from measure space $ m $ to $ n $ if:
\[
  \forall A \in \texttt{measurable\_sets}(n),\quad f^{-1}(A) \in \texttt{measurable\_sets}(m).
\]

In HOL4, the set of measurable functions is defined as:
\begin{hol}
  \begin{alltt}
    Definition measurable\_def :
    \(\!\!\!{\turn}\!\!\!\!\) \(\forall\)a b.
    measurable a b =
    \{f \,|\,
    f \(\in\) (space a \(\rightarrow\) space b) \(\land\)
    (\(\forall\)s. s \(\in\) subsets b \(\Rightarrow\) PREIMAGE f s \(\cap\) space a \(\in\) subsets a)\}
  \end{alltt}
\end{hol}


\subsection{Borel Sets and Measurable Functions}

A fundamental component of probability and integration in HOL4 is the notion of Borel sets and Borel-measurable functions. These are provided by two standard theories: \texttt{real\_borel}, which handles real-valued Borel sets and functions, and \texttt{borel}, which generalises these definitions to extended real values (\texttt{extreal}) and constructs Borel and Lebesgue measure spaces.

\paragraph{Borel $\sigma$-Algebra.}
In HOL4, the Borel $\sigma$-algebra on $ \mathbb{R} $, denoted \texttt{borel}, is defined as the smallest $\sigma$-algebra generated by open subsets of $ \mathbb{R} $:
\begin{hol}
  \begin{alltt}
    Definition borel :
    \(\!\!\!{\turn}\!\!\!\!\)
    borel = sigma \(\mathbb{U}\)(:real) \{s \,|\, open s\}
  \end{alltt}
\end{hol}


Equivalently, \texttt{borel} can be generated using various families of half spaces or intervals, such as $ \{ x \mid x \leq a \} $, $ \{ x \mid a < x < b \} $, and so on.

\paragraph{Extended Borel $\sigma$-Algebra.}
The extended real-valued Borel $\sigma$-algebra, \texttt{Borel}, is defined over the type \texttt{extreal}, which includes $ +\infty $ and $ -\infty $. It extends \texttt{borel} by including atomic sets containing infinities:

\begin{hol}
  \begin{alltt}
    Definition Borel :
    \(\!\!\!{\turn}\!\!\!\!\)
    Borel =
    (\(\mathbb{U}\)(:extreal),
    \{B' \,|\,
    \(\exists\)B S.
    B' = IMAGE Normal B \(\cup\) S \(\land\)
    B \(\in\) subsets borel \(\land\)
    S \(\in\) \{\(\emptyset\); \{\(-\infty\)\}; \{\(+\infty\)\}; \{\(-\infty\), \(+\infty\)\}\})
  \end{alltt}
\end{hol}


This construction allows measurable sets to include points at infinity, which is necessary when working with unbounded expectations or improper integrals.

\paragraph{Measurable Functions.}
A function $ f : X \to Y $ is measurable from a $\sigma$-algebra $ a $ to $ b $ if the preimage of any measurable set in $ b $ lies in the $\sigma$-algebra $ a $. This is formalised as:

\begin{hol}
  \begin{alltt}
    Definition measurable\_def :
    \(\!\!\!{\turn}\!\!\!\!\) \(\forall\)a b.
    measurable a b =
    \{f \,|\,
    f \(\in\) (space a \(\rightarrow\) space b) \(\land\)
    (\(\forall\)s. s \(\in\) subsets b \(\Rightarrow\) PREIMAGE f s \(\cap\) space a \(\in\) subsets a)\}
  \end{alltt}
\end{hol}

To simplify notation, HOL4 introduces:
\begin{hol}
\begin{verbatim}
val _ = overload_on ("borel_measurable", ``\a. measurable a borel``);
val _ = overload_on ("Borel_measurable", ``\a. measurable a Borel``);
\end{verbatim}
\end{hol}

Thus, a real-valued function is \texttt{borel\_measurable a} if measurable with respect to \texttt{borel}, and similarly for \texttt{Borel\_measurable a} for extended real-valued functions.

\paragraph{Measurability Closure Properties.}
Both \texttt{borel\_measurable} and \texttt{Borel\_measurable} are closed under arithmetic operations such as addition, subtraction, and multiplication. For example, if $ f, g \in \texttt{borel\_measurable a} $, then $ f + g $, $ f - g $, and $ f \cdot g $ are also in \texttt{borel\_measurable a}.

\begin{hol}
\begin{alltt}
Theorem in\_borel\_measurable\_add :
\(\!\!\!{\turn}\!\!\!\!\) \(\forall\)a f g h.
sigma\_algebra a \(\land\) f \(\in\) borel\_measurable a \(\land\)
g \(\in\) borel\_measurable a \(\land\)
(\(\forall\)x. x \(\in\) space a \(\Rightarrow\) h x = f x + g x) \(\Rightarrow\)
h \(\in\) borel\_measurable a
\end{alltt}
\end{hol}

\begin{hol}
\begin{alltt}
Theorem IN\_MEASURABLE\_BOREL\_ADD' :
\(\!\!\!{\turn}\!\!\!\!\) \(\forall\)a f g h.
sigma\_algebra a \(\land\) f \(\in\) Borel\_measurable a \(\land\)
g \(\in\) Borel\_measurable a \(\land\)
(\(\forall\)x. x \(\in\) space a \(\Rightarrow\) h x = f x + g x) \(\Rightarrow\)
h \(\in\) Borel\_measurable a
\end{alltt}
\end{hol}

\begin{hol}
\begin{alltt}
Theorem in\_borel\_measurable\_sub :
\(\!\!\!{\turn}\!\!\!\!\) \(\forall\)a f g h.
sigma\_algebra a \(\land\) f \(\in\) borel\_measurable a \(\land\)
g \(\in\) borel\_measurable a \(\land\)
(\(\forall\)x. x \(\in\) space a \(\Rightarrow\) h x = f x \({-}\) g x) \(\Rightarrow\)
h \(\in\) borel\_measurable a
\end{alltt}
\end{hol}

\begin{hol}
\begin{alltt}
Theorem IN\_MEASURABLE\_BOREL\_SUB' :
\(\!\!\!{\turn}\!\!\!\!\) \(\forall\)a f g h.
sigma\_algebra a \(\land\) f \(\in\) Borel\_measurable a \(\land\)
g \(\in\) Borel\_measurable a \(\land\)
(\(\forall\)x. x \(\in\) space a \(\Rightarrow\) h x = f x \({-}\) g x) \(\Rightarrow\)
h \(\in\) Borel\_measurable a
\end{alltt}
\end{hol}

\begin{hol}
\begin{alltt}
Theorem in\_borel\_measurable\_mul :
\(\!\!\!{\turn}\!\!\!\!\) \(\forall\)a f g h.
sigma\_algebra a \(\land\) f \(\in\) borel\_measurable a \(\land\)
g \(\in\) borel\_measurable a \(\land\)
(\(\forall\)x. x \(\in\) space a \(\Rightarrow\) h x = f x * g x) \(\Rightarrow\)
h \(\in\) borel\_measurable a
\end{alltt}
\end{hol}

\begin{hol}
\begin{alltt}
Theorem IN\_MEASURABLE\_BOREL\_TIMES' :
\(\!\!\!{\turn}\!\!\!\!\) \(\forall\)a f g h.
sigma\_algebra a \(\land\) f \(\in\) Borel\_measurable a \(\land\)
g \(\in\) Borel\_measurable a \(\land\)
(\(\forall\)x. x \(\in\) space a \(\Rightarrow\) h x = f x * g x) \(\Rightarrow\)
h \(\in\) Borel\_measurable a
\end{alltt}
\end{hol}


Infact, if a real-valued function is Borel-measurable, then its extension to extreal values is also Borel-measurable, and vice versa.

\begin{hol}
\begin{alltt}
Theorem IN\_MEASURABLE\_BOREL\_IMP\_BOREL' :
\(\!\!\!{\turn}\!\!\!\!\) \(\forall\)a f.
sigma\_algebra a \(\land\) f \(\in\) borel\_measurable a \(\Rightarrow\)
Normal \(\circ\) f \(\in\) Borel\_measurable a
\end{alltt}
\end{hol}

\begin{hol}
\begin{alltt}
Theorem in\_borel\_measurable\_from\_Borel :
\(\!\!\!{\turn}\!\!\!\!\) \(\forall\)a f.
sigma\_algebra a \(\land\) f \(\in\) Borel\_measurable a \(\Rightarrow\)
real \(\circ\) f \(\in\) borel\_measurable a
\end{alltt}
\end{hol}

\section{Lebesgue Integration Theory}

\section{Probability Theory}
