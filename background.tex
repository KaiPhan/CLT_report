\chapter[Background and Related Work]{Background and Related Work}
\label{chap:2}

% \section{Background and Related Work}
% \textbf{Purpose:} Position this work in the context of previous efforts and provide foundational concepts.
% \subsection{Overview of the Central Limit Theorem}
% \begin{itemize}
%     \item Explain the statement of the CLT and its generalizations (e.g., Lyapunov’s and Lindeberg’s approaches).
%     \item Focus on Lyapunov’s version as the basis of this proof:

%      \begin{equation}
%      \text{If } \frac{\Gamma_n}{s_n^3} \to 0 \text{ as } \space \space n \to \infty,
%      \end{equation}
%      then
%      \[
%      \frac{S_n}{s_n} \xrightarrow{d} \Phi.
%      \]
%      Define key terms: \( \Gamma_n, s_n, S_n \), and Lyapunov’s condition.
% \end{itemize}

The Central Limit Theory is one of the main theoretical results from the  Probability theory, a bridge from individual randomness to aggregate predictability. It marks the gradual approach of mathematical rigor and upgrading of the theorem used by addressing more and more difficult problems. Over centuries, different forms of CLTs have been derived, each of which has added to its understanding convergence and applications. This chapter will present the historical development of the CLT that describes its milestones and contributions. It will also examine the modern theorem proving tools and ways of formalization, focusing on Lyapunov's approach.

\section{Historical Background}
Let us just trace briefly the historical evolution of the CLT, critical milestones, key contributors, and the gradual refinement of its conditions and scope. The roots of the CLT go back to the dispute of the eighteenth century, specifically to the time Abraham de Moivre laid the foundation for this theorem. In 1733, de Moivre showed that the sum of a large number of independent and identically distributed random variables converges to the normal distribution-an elegant approximation to problems such as the rolling of dice or repeated coin flips \cite{de1733approximatio}. His work initiated the interplay between discrete distributions-such as the Binomial-and their continuous approximations, and laid the ground for developments to come.

In 1810, Pierre-Simon Laplace independently extended de Moivre's insights by proving a much more general form of this theorem. Under broader conditions, Laplace demonstrated convergence of the sum of independent random variables to a normal distribution, thereby establishing CLT as a universal principle. His work was used in applying the CLT to population statistics and measurement errors \cite{laplace1835oeuvres} and cemented its usefulness for aggregate phenomena analysis.

The 19th century would see the formalization of the central limit theorem (CLT) by Pafnuty Chebyshev, who also gave the theorem a rigorous foundation by the introduction of variance and expectation conditions. With that work, Chebyshev rendered the CLT mathematically precise, linking it to the developing area of probability theory \cite{chebyshev1890deux}. Building on this work was further advancing the generalization of the CLT by Alexander Lyapunov to sequences of independent random variables that are not identically distributed. His version of the theorem brought forth Lyapunov's condition, which contained definite criteria for convergence \cite{lyapunov1895pafnutii}.

In the early years of the 20th century, contributions by Lyapunov represented such a change in the evolution of the CLT. Unlike his predecessors, Lyapunov sought a “direct” and “elementary” proof. It would try to make the theorem internal relations simpler and relax the theorem's assumptions. His papers published in 1900 and 1901 reconstructed previous methods with a rigor of Weierstrass, and introduced new precision to the theorem by establishing explicit bounds on the normalized sum and the limiting normal distribution. This specificity answers Chebyshev's earlier demand for sharper probabilistic results although it should have simplified the proof through some clever but much deeper refinements of Poisson's arguments \cite{fischer2011history}.

Lyapunov’s goals extended beyond mere generalization; he aimed to illuminate the theorem’s internal structure and make it more accessible. Avoiding the complexities of characteristic functions and complex analysis, he focused on moment-based criteria, relying on real-valued tools like Taylor expansions, bounding techniques, and Lyapunov's inequality. Thus the balance here is theoretical rigor and practical applicability, two characteristics that still resonate well with modern efforts at formal verification as that undertaken in this thesis.

In contemporary times, further refinements of the CLT were provided by William Feller bringing it to discrete random variables \cite{feller1945}, and all subsequent developments that are probabilistic have been geared towards this common extension to stable distributions and higher dimensional data. These extensions demonstrate the theorem’s adaptability to various contexts, from stochastic processes to machine learning and data analysis.

Lyapunov’s version of the CLT represents a key milestone in this journey. It generalizes the theorem for independent random variables with finite moments so that it becomes an observed but pliant frame of reference for convergence. It is this balancing between abstractness and concreteness that is a characteristic of the whole development of probability during the modern period. Contemporaries of Lyapunov such as von Mises, Lévy, Cramér, Khinchin, and Kolmogorov likewise pursued abstract relations in probability theory while keeping an eye on practical relevance. This tension is paradigmatically present in Lyapunov's work, which illustrates the way in which results which are purely formal mathematics may nonetheless satisfy external criteria of applicability and utility. For more information about the history and the proofs of the Central Limit Theorem, see \cite{fischer2011history, adams2009life}.

This thesis is based on the generalization of CLT as expressed in Lyapunov's version. Because of its focus on real-valued methods and moment-based criteria, it avoids any complex tools like characteristic functions or complex analysis, hence being particularly suitable for formalization in HOL4. In this respect, Lyapunov's approach not only advances the mathematical generalization but also provides a practical framework to meet the rigor and accessibility requirements of automated reasoning systems by directly addressing convergence under relaxed conditions.

\section{Related Work}

Formalization of mathematical theorems as proofs has been a singular research focus in modern computational mathematics. The systems such as Isabelle/HOL \cite{isabelle}, Coq \cite{bertot2013interactive}, and HOL4 \cite{slind2008brief} have been heavily relied on over the years to encode and prove basic properties but offer several features and challenges.

One of the most significant achievements in this area was the formalization of CLT within the Isabelle/HOL proof assistant. This was made possible mainly by the rich library related to complex analysis in Isabelle/HOL, which provided the means to make use of characteristic functions within the proof. Characteristic functions are Fourier transforms of probability distributions that make convergence easier to analyze by drawing on their special mathematical properties. Central to the approach are two key steps: first, pointwise convergence of the characteristic function for the normalized sum of independent random variables to that of the standard normal distribution should be demonstrated; and second, by the application of the L\'evy Continuity Theorem, one obtains the desired convergence in distribution. These two steps are actually the core of the proof that eventually leads to the verification of CLT \cite{billingsley2017probability}.

The characteristic function approach is beautifully complemented by the infrastructure provided by Isabelle/HOL, which supports careful manipulation of complex-valued functions, differentiation, and limits. This thereby enables Isabelle/HOL to achieve the rigorous performance that is demanded by Fourier analysis and probability theory. Among the striking personalities behind this effort is Luke Serafin, who has produced a giant leap forward in the formalization of the CLT \cite{serafin2015formally}. Serafin's systematic encoding of the characteristic function method has ensured that it was entirely generic and modular. His contributions involved the formalization of key intermediate results such as the for sums of independent random variables and the fact that components of his work are reusable, enabling extension into applications that relate to probabilistic theorems.

The strength of Isabelle/HOL is to borrow from its excellent complex analysis library. The characteristic function-based proofs are formalized smoothly, and the modular and extensible framework allows researchers to always build on previous work, gradually increasing the scope of formalized mathematics. The CLT formalization within Isabelle/HOL is exemplary in showing how even the most intricate probabilistic results can be encoded rigorously with the help of proof assistants. But this also points to some of the challenges with this approach: it is very much based on Fourier transforms and complex-valued methods, an infrastructure which may not always be as easily available when moving to other proof systems. Also, characteristic functions are powerful but often complicate proofs in situations where simpler real-valued methods, such as those based on moments, would suffice.

The formalization of CLT in Isabelle/HOL serves as an inspiration and comparison for the research presented in this thesis. The Example of Isabelle/HOL shows that the CLT can be formalized with the help of characteristic functions but also brings into relief the limitations of such a strategy-if proof assistants like HOL4 lack libraries comparable for complex analysis with it. This thesis takes a different site following the approach with the theorem of Lyapunov, which involves real-valued moment conditions rather than characteristic functions. Following this path, the work in HOL4 demonstrates how easier and less convoluted tools enable formulating the CLT while considering the inherent challenges of formal verification.
