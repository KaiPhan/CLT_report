\chapter{Background and Related Work}
\label{chap:2}

\section{Background: The Central Limit Theorem}

The central limit theory is one of the fundamental results of the Probability theory, a theoretical connection from individual randomness to collective predictability. It states the empirical fact that the sum of large numbers of independent random variables converges to a normal distribution, regardless of the individual distributions, as long as some general assumptions hold. This theorem explains the universality of the normal distribution in nature and is the foundation for an extremely broad range of applications in statistics, physics, and computer science.

Historically, the beginnings of the CLT trace back to Abraham de Moivre, who showed in 1733 that the binomial distribution is approximated very well by the normal distribution if the number of trials is large \cite{fischer2011history}. Pierre-Simon Laplace later generalised this result and formalised it in terms of the Laplace–de Moivre theorem. Chebyshev and his students, especially Lyapunov and Markov, pushed the theory further by relaxing the identically distributed condition and replacing it with moment conditions. Lyapunov’s 1901 version introduced a now-famous condition on the third absolute moment, which is still the cornerstone for general versions of the CLT.

A full history of the development of the CLT can be found in Fischer’s \emph{A History of the Central Limit Theorem} \cite{fischer2011history}, which chronicles how the theorem evolved from numerical approximations to a fundamental limit law in probability. The evolution reflects a broader trend in mathematics: from combinatorial methods to analysis, and finally, to measure-theoretic and functional analytic foundations.

In this thesis, we focus on the Lyapunov form of the CLT, as presented in Chung’s textbook \cite{Chung:2001}. It assumes independence, but not identical distribution, and uses the Lindeberg replacement trick to incrementally replace variables with normal ones, bounding the error using Taylor's theorem. This analytic method avoids characteristic functions and relies only on real analysis and moment estimates. As a result, it is well-suited to formalisation in systems like HOL4, which are strong in real analysis but currently lack mature libraries for complex integration.

\section{Related Work}

Formal proofs of the CLT have previously been attempted in several proof assistants. The most notable example is the Isabelle/HOL formalisation by \cite{serafin2015formally}, which proves the CLT under the assumption of independent identically distributed (i.i.d.) random variables. That work follows the classical approach using characteristic functions and the Lévy Continuity Theorem.

While the Isabelle/HOL technique is elegant and mathematically sound, it relies on involved analysis and Fourier transforms, which are well-supported in Isabelle but not yet fully formalised in HOL4. HOL4 is lacking key theorems for complex-valued functions at present, so the characteristic function approach is unworkable here.

Moreover, the i.i.d. assumption limits the scope of the Isabelle formalisation. In contrast, this thesis formalises a strictly more general version of the CLT—Lyapunov’s form—which requires only independence and finiteness of variances and third absolute moments. The approach avoids characteristic functions entirely and instead uses the Lindeberg replacement method with Taylor expansion and asymptotic error bounds.

This makes the result both broader in applicability and better aligned with the current capabilities of HOL4’s real and measure-theoretic libraries. The infrastructure developed in this work—including handling of expectations, variances, random variable sequences, and Taylor bounds—may serve as a foundation for generalising beyond Lyapunov’s condition in future work.

To the best of our knowledge, this is the first mechanised proof of the Central Limit Theorem in HOL4 that goes beyond the i.i.d. case and formalises the full structure of the Lindeberg–Lyapunov strategy.
